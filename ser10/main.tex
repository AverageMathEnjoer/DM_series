%this is latex 2e
\documentclass[12pt,a4paper]{article}
\usepackage{amsmath,amsthm,amssymb}
\usepackage[mathscr]{eucal}
\usepackage[T2A]{fontenc}
\usepackage[utf8]{inputenc}
\usepackage[russian]{babel}
% \usepackage{wrapfig}
% \RequirePackage[pdftex]{graphicx}
% \usepackage{cells}

% \usepackage{pgf,tikz,pgfplots}
% \pgfplotsset{compat=1.9}
% \usepackage{mathrsfs}
% \usetikzlibrary{arrows}

\oddsidemargin= -5,4mm
\textwidth=170mm
\headheight=0pt
\headsep=0pt
\topmargin=10,4mm
\textheight=287mm



\pagestyle{empty}

\sloppy
\righthyphenmin=2
\exhyphenpenalty=10000

\def\ds{\displaystyle}
\def\ss{\scriptstyle}

% \cellsize=1.3ex

\def\q#1.{{\bf #1.}}
\def\N{{\mathbb N}}
\def\Z{\mathbb Z}
\def\Q{\mathbb Q}
\def\R{\mathbb R}
\def\Pp{\mathbb P}
\def\C{{\mathbb C}}
\def\N{{\mathbb N}}
\def\Cc{{\rm  C}}
\def\eps{{\varepsilon}}
\def\vec{\overrightarrow}
\def\ov{\overline}
\def\mmax{\mathop{\rm max}\limits}
% \def\Сhar{{\rm char}}
\def\sleg#1#2{\left(\frac{#1}{#2}\right)}
\def\ord{\rm ord}
\def\arc#1{\buildrel\,\,\smile\over{#1}}
\DeclareRobustCommand{\No}{\ifmmode{\nfss@text{\textnumero}}\else\textnumero\fi}

\def\amp{\mathbin\&}

\DeclareMathOperator{\Ker}{Ker}
\DeclareMathOperator{\Ima}{Im}
\DeclareMathOperator{\Int}{Int}
\DeclareMathOperator{\Cl}{Cl}

\newtheorem*{utv}{Утверждение}
\newtheorem*{thm}{Теорема}
\newtheorem*{rem}{Замечание}
\theoremstyle{definition}
\newtheorem*{ddefin}{Определение}
\newtheorem{defin}{Определение}

\begin{document}

% \small
% \footnotesize

% \renewcommand{\baselinestretch}{0.85}\selectfont
% \setbox0=\vbox{\lines3
%  _
% |_|_ _
% |_|_|_|
% }

\newser{{%
\centerline{\bf Серия  10. Орграфы (до 14.03.24)}
\medskip

\q1. В графе $2021$ вершина, и каждая вершина имеет степень $4$. На
каждом ребре этого графа поставили стрелочку. Докажите, что найдётся
вершина, в которую входит чётное число стрелок.

\q2. а) Докажите, что в любом графе без кратных рёбер есть две вершины одинаковой степени. (Петли и кратные рёбра запрещены.)

б) А верно ли, что в любом орграфе (без петель и кратных стрелок)
есть две вершины одинаковой исходящей степени?

\q3. Дан неполный ориентированный граф $G$ без кратных стрелок. При
добавлении любой стрелки (без появления кратных и встречных стрелок)
он становится сильно связным. Докажите, что граф $G$ является сильно
связным.

\q4. Пусть $G$ - сильно связный орграф с $\delta_+(G) \geq 2$. Докажите, что
существует такая вершина $v \in V(G)$, что орграф $G - v$
сильно связен

\q5. В орграфе $200$ вершин, из каждой вершины выходит хотя бы одна
стрелка и в каждую вершину входит хотя бы одна стрелка. Докажите,
что можно добавить не более $100$ новых стрелок так, чтобы этот орграф
стал сильно связным. (Между двумя вершинами может быть проведено
несколько стрелок.)

\q6. Орграф $D$ таков, что нериентированный граф $\underline{D}$ связен. В каждой
вершине $D$ входящих и исходящих стрелок поровну. Докажите, что $D$
эйлеров (то есть, имеет эйлеров цикл — ориентированный цикл, проходящей по каждой стрелке ровно один раз).
 
\q7. а) Докажите, что для любого $k \in \N$ существует сильно связный орграф $G$, не содержащий
четных циклов, все исходящие степени вершин которого равны $k$.

б) Докажите, что для любого $k \in \N$ существует орграф $G$ с $\delta_+(G) \geq k, \delta_-(G) \geq k$, не содержащий
четных циклов.

\q8. Докажите, что любой не сильно связный турнир (то есть, орграф, в
котором любые две вершины соединены ровно одной стрелкой) не менее
чем на 3 вершинах можно сделать сильно связным, изменив направление
одной стрелки.

\q9. Между волейбольными командами двух стран был проведен матчтурнир, в котором каждая команда сыграла ровно по одному разу со
всеми командами другой страны. При этом каждая команда выиграла
хотя бы одну встречу. Докажите, что найдутся четыре команды A, B, C
и D такие, что A выиграла у B, B выиграла у C, C выиграла у D, а D
выиграла у A. (Ничьих в волейболе не бывает).

\q10. Пусть $G$ - турнир на $n^2 + 1$ вершине. Его стрелки раскрашены
в два цвета так, что нет одноцветных циклов. Докажите, что в $G$ есть
одноцветный простой путь длины $n$.

\q11*. Пусть $G$ - турнир на $2020$ вершинах. Докажите, что в нем существует такой гамильтонов путь $a_1a_2 ... a_{2020}$, что его концы соединены
стрелкой $a_1a_{2020}$.

\q12*. Ребра связного графа $G$ разбили на два множества. Ребра из первого множества можно так ориетировать, что граф останется связным (при этом,
ребра второго множества остаются двусторонними). Ребра второго множества также можно ориентировать, не нарушив связности. Докажите,
что все ребра можно ориентировать так, чтобы получился сильно связный орграф.

\q13**. Сильно связный орграф $D$ таков, что $\delta_+(D) \geq 2, \delta_-(D) \geq 2$. Докажите, что орграф $D$ имеет
такой цикл $C$, что орграф $D - A(C)$ сильно связен.


\end{document}
