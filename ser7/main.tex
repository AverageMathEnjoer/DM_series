%this is latex 2e
\documentclass[12pt,a4paper]{article}
\usepackage{amsmath,amsthm,amssymb}
\usepackage[mathscr]{eucal}
\usepackage[T2A]{fontenc}
\usepackage[utf8]{inputenc}
\usepackage[russian]{babel}
% \usepackage{wrapfig}
% \RequirePackage[pdftex]{graphicx}
% \usepackage{cells}

% \usepackage{pgf,tikz,pgfplots}
% \pgfplotsset{compat=1.9}
% \usepackage{mathrsfs}
% \usetikzlibrary{arrows}

\oddsidemargin= -5,4mm
\textwidth=170mm
\headheight=0pt
\headsep=0pt
\topmargin=10,4mm
\textheight=287mm


\pagestyle{empty}

\sloppy
\righthyphenmin=2
\exhyphenpenalty=10000

\def\ds{\displaystyle}
\def\ss{\scriptstyle}

% \cellsize=1.3ex

\def\q#1.{{\bf #1.}}
\def\N{\mathbb N}
\def\Z{\mathbb Z}
\def\Q{\mathbb Q}
\def\R{\mathbb R}
\def\Pp{\mathbb P}
\def\C{{\mathbb C}}
\def\Cc{{\rm  C}}
\def\eps{{\varepsilon}}
\def\vec{\overrightarrow}
\def\ov{\overline}
\def\mmax{\mathop{\rm max}\limits}
% \def\Сhar{{\rm char}}
\def\sleg#1#2{\left(\frac{#1}{#2}\right)}
\def\ord{\rm ord}
\def\arc#1{\buildrel\,\,\smile\over{#1}}
\DeclareRobustCommand{\No}{\ifmmode{\nfss@text{\textnumero}}\else\textnumero\fi}

\def\amp{\mathbin\&}

\DeclareMathOperator{\Ker}{Ker}
\DeclareMathOperator{\Ima}{Im}
\DeclareMathOperator{\Int}{Int}
\DeclareMathOperator{\Cl}{Cl}

\newtheorem*{utv}{Утверждение}
\newtheorem*{thm}{Теорема}
\newtheorem*{rem}{Замечание}
\theoremstyle{definition}
\newtheorem*{ddefin}{Определение}
\newtheorem{defin}{Определение}

\begin{document}

% \small
% \footnotesize

% \renewcommand{\baselinestretch}{0.85}\selectfont
% \setbox0=\vbox{\lines3
%  _
% |_|_ _
% |_|_|_|
% }

\newser{{%
\centerline{\bf Серия 7. }
\medskip

\q1. В королевстве живут рыцари. Любые два из них либо враждуют
(и такие среди них есть!), либо дружат, либо друг к другу безразличны.
Друг врага рыцаря — враг этого рыцаря. Докажите, что хотя бы у одного
рыцаря врагов больше, чем друзей.

\q2. Ребра полного графа $K_{100}$ раскрашены в синий и зеленый цвета.
Для любых пяти городов оказалось, что среди всех ребер между ними не
более четырех синих. Докажите, что во всем графе синих ребер меньше,
чем зеленых.

\q3. Пусть $G$ — связный граф, $W \subset V(G)$. Докажите, что два утверждения равносильны.

1) Существует остовное дерево, в котором все вершины множества $W$
являются висячими.

2) Для любого множества вершин $U \subseteq W$ граф $G -U$ связен

\q4. В регулярном графе $G$ степени $d$ нечетное число вершин. Докажите, что $\chi'(G) = d + 1$

\q5.  Пусть $G$ — двудольный граф

а) Докажите, что $G$ имеет регулярный двудольный надграф степени
$\Delta(G)$ (добавлять можно как вершины, так и рёбра).

б) Докажите, что $\chi'(G) = \Delta(G)$ (с помощью пункта а и теорем о паросочетаниях).


\q6. Дан граф $G$ c $e(G) > nv(G)$, где $n \in \N$. Докажите, что этот граф
не является $(n + 1)$-редуцируемым.

\q7. Дан граф, степени всех вершин которого равны $4$. Его вершины
покрашены в три цвета. Докажите, что есть цикл, вершины которого
покрашены не более, чем в два цвета.

\q8. На вечеринке гость считается застенчивым, если у него не более
трех знакомых. Оказалось, что у каждого гостя не менее трех застенчивых знакомых. Докажите, что все гости — застенчивые.

\q9. а) Докажите, что можно удалить не более $\frac{1}{k}$
рёбер из графа так,
чтобы получился граф хроматического числа не более $k$.

б) Докажите, что любой граф $G$ имеет такой подграф $G'$ с $\chi(G') \leq k$,
что для любой вершины $v$ выполнено $d_{G'}(v) \geq \frac{k-1}{k}d_{G}(v)$.


\end{document}
