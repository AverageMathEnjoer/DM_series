%this is latex 2e
\documentclass[12pt,a4paper]{article}
\usepackage{amsmath,amsthm,amssymb}
\usepackage[mathscr]{eucal}
\usepackage[T2A]{fontenc}
\usepackage[utf8]{inputenc}
\usepackage[russian]{babel}
% \usepackage{wrapfig}
% \RequirePackage[pdftex]{graphicx}
% \usepackage{cells}

% \usepackage{pgf,tikz,pgfplots}
% \pgfplotsset{compat=1.9}
% \usepackage{mathrsfs}
% \usetikzlibrary{arrows}

\oddsidemargin= -5,4mm
\textwidth=170mm
\headheight=0pt
\headsep=0pt
\topmargin=10,4mm
\textheight=287mm


\pagestyle{empty}

\sloppy
\righthyphenmin=2
\exhyphenpenalty=10000

\def\ds{\displaystyle}
\def\ss{\scriptstyle}

% \cellsize=1.3ex

\def\q#1.{{\bf #1.}}
\def\N{{\mathbb N}}
\def\Z{\mathbb Z}
\def\Q{\mathbb Q}
\def\R{\mathbb R}
\def\Pp{\mathbb P}
\def\C{{\mathbb C}}
\def\N{{\mathbb N}}
\def\Cc{{\rm  C}}
\def\eps{{\varepsilon}}
\def\vec{\overrightarrow}
\def\ov{\overline}
\def\mmax{\mathop{\rm max}\limits}
% \def\Сhar{{\rm char}}
\def\sleg#1#2{\left(\frac{#1}{#2}\right)}
\def\ord{\rm ord}
\def\arc#1{\buildrel\,\,\smile\over{#1}}
\DeclareRobustCommand{\No}{\ifmmode{\nfss@text{\textnumero}}\else\textnumero\fi}

\def\amp{\mathbin\&}

\DeclareMathOperator{\Ker}{Ker}
\DeclareMathOperator{\Ima}{Im}
\DeclareMathOperator{\Int}{Int}
\DeclareMathOperator{\Cl}{Cl}

\newtheorem*{utv}{Утверждение}
\newtheorem*{thm}{Теорема}
\newtheorem*{rem}{Замечание}
\theoremstyle{definition}
\newtheorem*{ddefin}{Определение}
\newtheorem{defin}{Определение}

\begin{document}

% \small
% \footnotesize

% \renewcommand{\baselinestretch}{0.85}\selectfont
% \setbox0=\vbox{\lines3
%  _
% |_|_ _
% |_|_|_|
% }

\newser{{%
\centerline{\bf Серия  8. Планарность и немного стереометрии (до 21.02.24)}
\medskip

\q1. Докажите, что для любого $n \in \N$ существует двусвязный планарный граф $G$ с $v(G) > n$, который имеет два неизоморфных плоских
изображения.

\q2. Докажите, что для любого $n \in \N$ существует планарный граф $G$ с $v(G) > n$, что графы $G$ и $G^*$ изоморфны.

\q3. Докажите, что для любого $n \in \N$ существует планарный граф, существует планарный граф, такой что $v(G) = n, e(G) =
3n - 6$.

\q4. Докажите, что любой выпуклый многогранник имеет либо вершину степени 3, либо грань-треугольник. (Выпуклый многогранник —
трёхсвязный граф без петель и кратных рёбер, который можно изобразить на сфере без пересечений рёбер.)

\q5. Многогранник называется правильным, если он выпуклый, в каждой вершине сходится одно и то же число рёбер, а его грани — одинаковые
правильные многоугольники.

а) Пусть у правильного многогранника в каждой вершине сходится $k$
рёбер, а грань — это $s$-угольник. Докажите, что $(k - 2)(s - 2) < 4$.

б) Перечислите все правильные многогранники и докажите, что других нет.

\q6. В графе $2000$ вершин, все они имеют степень $7$. Докажите, что в
этом графе можно выбрать $777$ ребер, не имеющих общих концов.

\q7. Все грани многогранника являются треугольниками. Каждая грань
окрашена в черный или белый цвет так, что количество ребер, по которым граничат одноцветные грани, минимально. Пусть $a$ и $b$ — количества
белых и черных граней, соответственно. Докажите, что $a \leq 1,5b$.

\q8. Докажите, что грани плоского графа можно правильным образом
покрасить в 2 цвета тогда и только тогда, когда степени всех вершин
четны.
 
\q9. Пусть $G$ - связный плоский граф без мостов, кратных ребер и петель и $|V (G)| \geq 3$.
Докажите, что $G$ - двухсвязен $\Longleftrightarrow$ $G^*$
- двухсвязен.

\q10. Граф называется \textbf{внешне планарным}, если он имеет изображение, в котором
каждая вершина лежит на границе внешней грани. Докажите, что граф является
внешне планарным, если и только если он не содержит в качестве подграфа ни подразбиение $K_4$, ни подразбиение $K_{2,3}$.

\q11*. В триангуляции все вершины имеют степень хотя бы 5,
причем никакие две вершины степени 5 не смежны. Докажите, что есть треугольник
с вершинами степеней 5, 6 и 6.

\q12*. Граф $G$ при удалении любой вершины становится планарным. Докажите, что
$\chi(G) \leq 5$.(теорему о $4$-ех красках можно использовать только с вместе доказательством)

\q13*. Изобразите на торе без пересечений во внутренних точках:

а) $K_7$

б) $K_{4,4}$

\q14*. Докажите, что на торе без пересечений во внутренних точках можно изобразить граф $G$, полученный из цикла на $2k$ вершинах$(k \geq 2)$, путём добавления рёбер,
соединяющих противоположные пары вершин цикла.

\end{document}
